\textbf{Ativismo digital hoje} \lipsum[1]

\textbf{Claudio Penteado} é cientista social e doutor em Ciências Sociais pela \textsc{puc/sp}. É pesquisador do Núcleo de Estudos em Arte, Mídia e Política (\textsc{neamp -- puc/sp}), no qual participa de pesquisas de projetos temáticos \textsc{fapesp} sobre política e novas tecnologias.

\textbf{Rosemary Segurado} é doutora em Ciências Sociais pela \textsc{puc/sp} e pós-doutora em Comunicação Política pela Universidade Rey Juan Carlos de Madrid (2008). Atualmente é pesquisadora da \textsc{puc/sp} e coordenadora do Curso Mídia, Política e Sociedade da Fundação Escola de Sociologia e Política de São Paulo. Pesquisadora do Núcleo de Estudos em Arte, Mídia e Política da (\textsc{neamp -- puc/sp}) e editora da \textit{Revista Aurora}.

\textbf{Sergio Amadeu da Silveira} é sociólogo e doutor em Ciência Política pela Universidade de São Paulo (2005). É professor associado da Universidade Federal do \textsc{abc} (\textsc{ufabc}). Publicou \textit{Comunicação digital e a construção dos commons} (Perseu Abramo, 2007), \textit{Tudo sobre tod@s: Redes digitais, privacidade e venda de dados pessoais} (Edições Sesc, 2017), \textit{Democracia e os códigos invisíveis: como os algoritmos estão modulando comportamentos e escolhas políticas} (Edições Sesc, 2019), entre outros.

