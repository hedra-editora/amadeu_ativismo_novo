\textbf{Ativismo digital hoje} reúne nove textos que refletem, de diferentes óticas, sobre a influência das redes sociais na política e na cultura atualmente: ciberfeminismo, democracia digital, políticas online, ativismo online, cibervigilância, conflitos nas redes sociais, governo aberto, governança da Internet e cultura digital são alguns dos temas explorados.
Pela abrangência do assunto, os artigos são divididos em três eixos: ciberpolítica, com enfoque na interface digital da política; ciberativismo, que percorre as alterações no campo do ativismo ocorridas desde a década de 1990; e cibercultura, que explora a emergência de práticas culturais e a expressão de subjetividades em consonância com as novas práticas comunicacionais do meio digital.

\textbf{Claudio Penteado} é cientista social e doutor em Ciências Sociais pela \textsc{puc--sp}. É pesquisador do Núcleo de Estudos em Arte, Mídia e Política (\textsc{neamp/\,puc--sp}), no qual participa de pesquisas de projetos temáticos \textsc{fapesp} sobre política e novas tecnologias.

\textbf{Rosemary Segurado} é doutora em Ciências Sociais pela \textsc{puc--sp} e pós-doutora em Comunicação Política pela Universidade Rey Juan Carlos de Madrid. Atualmente é pesquisadora da \textsc{puc--sp} e coordenadora do curso Mídia, Política e Sociedade, da Fundação Escola de Sociologia e Política de São Paulo. 
E também pesquisadora do Núcleo de Estudos em Arte, Mídia e Política da (\textsc{neamp/\,puc--sp}) e editora da \textit{Revista Aurora}.

\textbf{Sérgio Amadeu da Silveira} é sociólogo e doutor em Ciência Política pela Universidade de São Paulo. É professor associado da Universidade Federal do \textsc{abc} (\textsc{ufabc}). Publicou \textit{Comunicação digital e a construção dos commons} (2007), \textit{Tudo sobre tod@s: Redes digitais, privacidade e venda de dados pessoais} (2017), \textit{Democracia e os códigos invisíveis: como os algoritmos estão modulando comportamentos e escolhas políticas} (2019), entre outros.

